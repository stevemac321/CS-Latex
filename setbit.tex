\documentclass{article}
\usepackage{amsmath}
\begin{document}

\section*{Step-by-Step Process}

\subsection*{1. Create the Mask}
We start with \(1\) and shift it left by \texttt{pos} positions.
\[
1U \ll 6 \quad \text{(shifting 1 left by 6 positions)}
\]
\[
\begin{array}{c}
00000001 \quad \text{(1 in binary)} \\
\ll 6 \\
\hline
01000000 \quad \text{(This is the mask with a 1 at position 6)}
\end{array}
\]

\subsection*{2. Invert the Mask}
Next, we invert the mask using the ones complement operator (\(\sim\)).
\[
\sim(01000000)
\]
\[
\begin{array}{c}
01000000 \quad \text{(Original mask)} \\
\sim \\
\hline
10111111 \quad \text{(Inverted mask)}
\end{array}
\]

\subsection*{3. Apply the Mask to Clear the Bit}
Now, we use the bitwise AND assignment (\&=) operator with the inverted mask to clear the bit at position 6 in \texttt{vec}.
\[
\texttt{vec} \&= 10111111
\]
\[
\begin{array}{c}
00001111 \quad \texttt{(vec)} \\
\& \\
10111111 \quad \texttt{(Inverted mask)} \\
\hline
00001111 \quad \texttt{(Result after clearing the bit at position 6)}
\end{array}
\]

\section*{Summary}
\begin{itemize}
    \item \textbf{Mask Creation}: \(1U \ll 6\) results in \(01000000\).
    \item \textbf{Mask Inversion}: \(\sim(01000000)\) results in \(10111111\).
    \item \textbf{Bit Clearing}: \(\texttt{vec} \&= 10111111\) results in \(00001111\).
\end{itemize}

\end{document}
