\documentclass{article}
\usepackage{amsmath}

\title{Understanding the Base Case in Induction and Recursion}
\author{}
\date{}

\begin{document}

\maketitle

\section{Understanding the Base Case in Mathematical Induction}

In mathematical induction:

\begin{itemize}
    \item \textbf{Inductive Hypothesis}: This is an assumption that the statement is true for some arbitrary \( n \). You use this assumption to prove that the statement must also be true for \( n + 1 \).
    \item \textbf{Base Case (Basis Case)}: This is the simplest instance of the problem, where you directly verify that the statement holds true. It often corresponds to the smallest value for which the statement is supposed to be true (e.g., \( n = 0 \) or \( n = 1 \)).
\end{itemize}

In \textbf{Example 2.4} from \textit{Foundations}:

\begin{itemize}
    \item The \textbf{inductive hypothesis} is that the sum of powers of 2 from \( i = 0 \) to \( n \) equals \( 2^{n+1} - 1 \).
    \item The \textbf{base case} is where \( n = 0 \). Here, you check directly whether the formula holds: 
    \[
    \sum_{i=0}^{0} 2^i = 2^{0+1} - 1.
    \]
\end{itemize}

\section{Connection to Recursion and Iteration}

In \textbf{recursion}:

\begin{itemize}
    \item The \textbf{base case} is the smallest instance where the recursion stops, and it's directly comparable to the base case in an inductive proof.
    \item The recursive function relies on the inductive hypothesis as it breaks down the problem into smaller subproblems, eventually reaching the base case, where it begins to "unwind" the recursive calls.
\end{itemize}

In \textbf{iteration}:

\begin{itemize}
    \item The base case isn't explicitly defined as it is in recursion or induction. Instead, the process typically starts with an initial condition and continues iterating until a termination condition is met.
\end{itemize}

\section{Your Confusion with CLRS and Merge Sort}

In \textbf{CLRS}:

\begin{itemize}
    \item The \textbf{inductive proof} of the Merge Sort algorithm involves proving that the algorithm correctly sorts any array of size \( n \) by assuming it correctly sorts arrays of smaller sizes (the inductive hypothesis).
    \item The \textbf{base case} for Merge Sort in CLRS is when the array has only one element (\( n = 1 \)). This base case corresponds to the point where the recursion stops because a single element is trivially sorted.
\end{itemize}

\section{The Source of Your Confusion}

The confusion arises because in \textbf{induction} and \textbf{recursion}, the term "base case" is used similarly but operates differently in context.

\begin{itemize}
    \item In \textbf{induction}, the base case is where you start your proof, verifying that the statement holds for the smallest value of \( n \).
    \item In \textbf{recursion}, the base case is where the recursive process stops, and you start solving the problem from the smallest possible subproblem.
\end{itemize}

In the case of the Merge Sort proof in CLRS, the "base case" refers to the situation where the array has only one element, and it's already sorted. This is analogous to proving the base case in an inductive proof.

\section{Summary of Understanding}

\begin{itemize}
    \item \textbf{Inductive Hypothesis}: Assumes the correctness for a given \( n \).
    \item \textbf{Base Case in Induction}: Verifies the statement for the smallest \( n \).
    \item \textbf{Base Case in Recursion}: Stops the recursion and starts returning results.
\end{itemize}

Your confusion is understandable because the base case in recursion is more of an operational point where the function stops calling itself, while in induction, it's a conceptual starting point for proving the correctness of a statement. The two concepts are closely related but used in slightly different ways.

\end{document}
