\documentclass{article}
\usepackage{amsmath}
\begin{document}

\section*{Understanding Merge Sort: Recursion and Merging}

\subsection*{1. Partitioning (Breaking into Subarrays)}
\textbf{Recursion Begins with Division:} The merge sort algorithm starts by recursively dividing the array into smaller subarrays. This division continues until each subarray contains only one element. The recursive \texttt{merge-sort} function calls itself to handle the left and right halves of the array.

\textbf{Recursive Calls Create the Subarrays:} These recursive calls keep breaking down the array into smaller parts:
\begin{itemize}
    \item For example, an array \(\{38, 27, 43, 3, 9, 82, 10\}\) is first split into \(\{38, 27, 43\}\) and \(\{3, 9, 82, 10\}\).
    \item \(\{38, 27, 43\}\) is then split into \(\{38\}\) and \(\{27, 43\}\).
    \item \(\{27, 43\}\) is further split into \(\{27\}\) and \(\{43\}\).
\end{itemize}

\subsection*{2. Recursion "Bottoms Out"}
\textbf{Base Case (Single Element Subarrays):} The recursion bottoms out when the array is split into single-element subarrays. At this point, the recursive calls no longer divide the array, as a single element is inherently sorted.

\subsection*{3. Merging Begins}
\textbf{Merge Work Starts with the Smallest Subarrays:} Once the recursion bottoms out, the merge process begins. The merge operation starts with the smallest subarrays—those containing only one element each.
\begin{itemize}
    \item For example, \(\{27\}\) and \(\{43\}\) are merged into \(\{27, 43\}\).
\end{itemize}

\textbf{Recursive Unwinding:} As the recursion stack unwinds, the merge function is called to combine the results of the previous merges:
\begin{itemize}
    \item \(\{27, 43\}\) is merged with \(\{38\}\) to form \(\{27, 38, 43\}\).
    \item On the other side, \(\{3\}\) and \(\{9\}\) merge to form \(\{3, 9\}\), which then merges with \(\{10, 82\}\) to form \(\{3, 9, 10, 82\}\).
\end{itemize}

\subsection*{4. Final Merge (Combining the Halves)}
\textbf{Combining Larger Subarrays:} The merging continues up the recursion stack, combining larger and larger subarrays. The final step merges the two halves of the array:
\begin{itemize}
    \item The subarrays \(\{27, 38, 43\}\) and \(\{3, 9, 10, 82\}\) are merged to form the fully sorted array \(\{3, 9, 10, 27, 38, 43, 82\}\).
\end{itemize}

\subsection*{5. Selection Process During Merging}
\textbf{Selecting Elements for Merging:} During the merging process, elements from the left and right subarrays (\texttt{L} and \texttt{R}) are compared, and the smallest element is selected and copied back into the original array \texttt{A}. This is like sorting face-up cards by repeatedly picking the smallest from two piles.

\textbf{Final Sorted Array:} The process culminates in the final merge step, where the last two large subarrays are merged, producing the sorted array.

\subsection*{Summary of the Process}
\textbf{Partitioning/Division:} This occurs first, with the array being recursively split until each subarray has only one element.

\textbf{Recursion Bottoms Out:} The base case is reached when subarrays cannot be split further.

\textbf{Merging:} The actual sorting happens during the merge phase, which begins as the recursion unwinds. Small subarrays are merged first, progressively leading to the merging of larger subarrays.

\textbf{Final Merge:} The last step combines the two halves of the array into a fully sorted array.

\end{document}
