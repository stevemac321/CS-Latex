\documentclass{article}
\usepackage{amsmath}
\begin{document}

\section*{Code Explanation}

\begin{verbatim}
for i in 0..bit_count {
    let a_bit = (a >> i) & T::from(1);
    let b_bit = (b >> i) & T::from(1);

    let sum_bit = a_bit ^ b_bit ^ c_in;
    carry = (a_bit & b_bit) | (b_bit & c_in) | (c_in & a_bit);

    sum |= sum_bit << i;
    c_in = carry;
}
\end{verbatim}

\section*{Step-by-Step Breakdown}

\subsection*{1. Loop Initialization}
The loop runs from \(i = 0\) to \(i < \text{bit\_count}\), iterating over each bit position.

\subsection*{2. Extracting Bits}
\begin{itemize}
    \item \texttt{let a\_bit = (a >> i) \& T::from(1);}
    \begin{itemize}
        \item \((a >> i)\) shifts the bits of \(a\) to the right by \(i\) positions.
        \item \(\& T::from(1)\) isolates the least significant bit (LSB) after the shift, effectively extracting the bit at position \(i\) from \(a\).
    \end{itemize}
    \item \texttt{let b\_bit = (b >> i) \& T::from(1);}
    \begin{itemize}
        \item Similarly, this extracts the bit at position \(i\) from \(b\).
    \end{itemize}
\end{itemize}

\subsection*{3. Calculating the Sum Bit}
\begin{itemize}
    \item \texttt{let sum\_bit = a\_bit \(\oplus\) b\_bit \(\oplus\) c\_in;}
    \begin{itemize}
        \item \(\oplus\) is the bitwise XOR operator.
        \item The sum bit is calculated using the XOR of \(a\_bit\), \(b\_bit\), and the carry-in (\(c\_in\)). This is because XOR of two bits gives the sum without carry.
    \end{itemize}
\end{itemize}

\subsection*{4. Calculating the Carry}
\begin{itemize}
    \item \texttt{carry = (a\_bit \(\land\) b\_bit) | (b\_bit \(\land\) c\_in) | (c\_in \(\land\) a\_bit);}
    \begin{itemize}
        \item \(\land\) is the bitwise AND operator.
        \item \(|\) is the bitwise OR operator.
        \item The carry is calculated using the AND of pairs of bits and the carry-in. This ensures that the carry is set if any two of the three bits (\(a\_bit\), \(b\_bit\), \(c\_in\)) are \(1\).
    \end{itemize}
\end{itemize}

\subsection*{5. Updating the Sum}
\begin{itemize}
    \item \texttt{sum |= sum\_bit << i;}
    \begin{itemize}
        \item \texttt{sum\_bit << i} shifts the sum bit to the correct position.
        \item \(|=\) is the bitwise OR assignment operator, which updates the \texttt{sum} by setting the bit at position \(i\) to \texttt{sum\_bit}.
    \end{itemize}
\end{itemize}

\subsection*{6. Updating the Carry-In}
\begin{itemize}
    \item \texttt{c\_in = carry;}
    \begin{itemize}
        \item The carry-out from the current bit position becomes the carry-in for the next bit position.
    \end{itemize}
\end{itemize}

\section*{Example}
Let's use some example values to illustrate:

\begin{itemize}
    \item \(a\_bit = 1\)
    \item \(b\_bit = 0\)
    \item \(c\_in = 1\)
\end{itemize}

For the XOR operation:
\begin{enumerate}
    \item \(a\_bit \oplus b\_bit\) results in \(1 \oplus 0 = 1\).
    \item \(1 \oplus c\_in\) results in \(1 \oplus 1 = 0\).
\end{enumerate}

For the OR operation:
\begin{enumerate}
    \item \((a\_bit \land b\_bit)\) results in \(1 \land 0 = 0\).
    \item \((b\_bit \land c\_in)\) results in \(0 \land 1 = 0\).
    \item \((c\_in \land a\_bit)\) results in \(1 \land 1 = 1\).
    \item Combining these with OR: \(0 | 0 | 1 = 1\).
\end{enumerate}

\end{document}
