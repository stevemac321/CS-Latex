\documentclass{article}

\begin{document}

\textbf{Key Points:}

1. \textbf{Linear Function ($c_1$):}
   - The cost $c_1$ associated with the outer \texttt{for} loop is linear in $n$ because the loop runs $n-1$ times.
   - This means the cost function for the outer loop is proportional to $n$, often expressed as $c_1 \cdot (n-1)$.

2. \textbf{Varying Inner Loop ($\Sigma$):}
   - For the inner loop, which runs a different number of times depending on the value of $j$, you can't simply multiply by $n$ as you would with the outer loop. Instead, you need to sum up the costs across all iterations of the inner loop.
   - Summation notation $\Sigma$ is used to represent the total cost over all iterations of the inner loop, which reflects the fact that the "window" (or the number of iterations) changes with each step of $j$.

   For example, if the inner loop starts at $j$ and runs until $0$, the number of times the inner loop runs is $j$, and the summation $\Sigma$ would account for all such $j$ values from $2$ to $n$.

\textbf{Example in Insertion Sort:}

- \textbf{Outer Loop (Linear):}
  - The outer loop runs $n-1$ times, so the cost is linear in $n$.
  - This is captured by $c_1 \cdot (n-1)$.

- \textbf{Inner Loop (Summation):}
  - The inner loop runs $j$ times for each $j$, and you sum over all $j$ values.
  - The total cost is represented as a summation: $\sum_{j=2}^{n} t_j$, where $t_j$ depends on the specific operations within the inner loop.

\textbf{Summation Notation:}

- \textbf{Why Use $\Sigma$?}
  - When analyzing loops that do not run a fixed number of times (like the inner loop in insertion sort), summation notation accurately reflects the total number of operations.
  - It captures the idea that the inner loop's runtime is dependent on the current state of the outer loop, not just on $n$.

\textbf{Summary:}

- The cost $c_1$ for the outer loop is linear because it scales with $n$.
- The inner loop's cost cannot be simply represented as a multiple of $n$; instead, it requires summation notation to account for the changing number of iterations as $j$ progresses.

\end{document}
