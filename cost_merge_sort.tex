\documentclass{article}
\usepackage{amsmath}
\begin{document}

\section*{Merge Sort Time Complexity Analysis}

\begin{itemize}
    \item Merge sort has a time complexity of \(O(n \log n)\).
    \item The outer function that the user calls drives the recursion.
    \item The recursive part contributes the \(\log n\) factor.
    \begin{itemize}
        \item For example, if \(n = 8\):
        \begin{itemize}
            \item The first recursion breaks 8 into 2 parts of 4.
            \item The next recursion breaks each 4 into 2 parts of 2.
            \item The recursion bottoms out when the subarrays are of size one.
        \end{itemize}
    \end{itemize}
    \item When the stack unwinds, the merge work is done:
    \begin{itemize}
        \item Merging the 1's into 2's.
        \item Then the 2's into 4's.
        \item Finally, the 4's into the final 8.
    \end{itemize}
    \item The merge operation contributes the \(n\) factor.
    \item The recursive iterations are counted, representing the number of "deployments."
    \item For example, you don't count \(1 + 4 + 4 + 2 + 2 + 2 + 2 + 1 + 1 + 1 + 1 + 1 + 1 + 1 + 1\) bottom, then \(2 + 2 + 2 + 2\), \(4 + 4\) to 8.
    \item These counts are not factored into Big O notation because they are constants.
    \item In Big O notation, constants and lower-order terms are ignored, focusing on the dominant term, which is \(O(n \log n)\).
\end{itemize}

\end{document}
