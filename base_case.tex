\documentclass{article}
\usepackage{amsmath}

\title{Understanding the Role of the "For" Statement in Recursion}
\author{}
\date{}

\begin{document}

\maketitle

\section{Clarifying the "For" Statement in Recursion}

The recurrence relation discussed is:
\[
C(n) = C(n-1) + n \quad \text{for } n \geq 2 \quad \text{with } C(1) = 1
\]

This defines how the recursive process behaves and when it terminates. Let's break this down.

\subsection{The "For \( n \geq 2 \)" Statement}

- This part of the recurrence relation specifies the condition under which the recursion continues. 
- It tells you that as long as \( n \) is 2 or greater, the recursive formula \( C(n) = C(n-1) + n \) applies. 
- This means that the recursive process keeps going, breaking the problem down further.

\subsection{Bottoming Out}

- The recursion "bottoms out" when \( n < 2 \). 
- Specifically, when \( n = 1 \), the recursion stops because you've reached the base case \( C(1) = 1 \). 
- This is the point where the problem is so small that it can be solved directly without further recursion.

\subsection{Base Case \( C(1) = 1 \)}

- The base case defines the smallest possible problem size and the cost associated with solving it. 
- In this case, the work of the base case is 1 unit of cost, meaning that when \( n = 1 \), the function doesn't recurse any further and simply returns 1.

\subsection{Summary}

This understanding clarifies that:
\begin{itemize}
    \item \textbf{Recursion Continues}: As long as \( n \geq 2 \), the recursive process will continue.
    \item \textbf{Recursion Stops}: When \( n = 1 \), the recursion stops, and the base case cost \( C(1) = 1 \) is returned.
    \item \textbf{Work at the Base Case}: The work done at the base case is minimal (1 unit of cost), serving as the foundation for the entire recursive process.
\end{itemize}

You've captured the essence of how the "for" statement defines the conditions for recursion and how the base case represents the point where the recursion ceases and begins to unwind.

\end{document}
